Content-Type: text/x-zim-wiki
Wiki-Format: zim 0.4
Creation-Date: 2015-09-10T15:22:24+04:30

====== Latex ======

Created پنجشنبه 10 سپتامبر 2015

== install problem ==
//sudo dpkg --purge --force-depends texlive-lang-other texlive-latex-extra tex-common texlive-fonts-recommended texlive-pictures texlive-metapost//
//sudo apt-get install -f texlive-lang-other texlive-latex-extra tex-common texlive-fonts-recommended texlive-pictures texlive-metapost//
//sudo mktexlsr//
//sudo updmap-sys// 



\renewcommand{\familydefault}{\sfdefault}
This changes the default font family to sans-serif.


== To create it with geometry is easy ==
\usepackage[a4paper, total={6in, 8in}]{geometry}

== set margin ==
\usepackage[margin=0.5in]{geometry}

\vspace*{3\baselineskip}

\usepackage{geometry}
\geometry{
a4paper,
total={170mm,257mm},
left=20mm,
top=20mm,
}


== insert picture in tabular ==
**\documentclass{article}**
**\usepackage{graphicx}**
**\begin{document}**
**\centering**
**   \begin{tabular}{c c}**
**  \includegraphics[scale=.41]{pic/1b.png}  &  \includegraphics[scale=.41]{pic/1a.png} \\**
**\end{tabular}**
**  \caption{A boat.}**
**  \label{fig:boat1}**
**\end{figure}**
**Figure \ref{fig:boat1} shows a boat.**
**\end{document}**

== How to center a specific caption? ==
\usepackage{caption}
\captionsetup{justification=centering}
or
\captionsetup{justification=centering,margin=2cm}

== page numbering ==
\pagenumbering{gobble}

\vspace*{3\baselineskip}

== insert picture in tabular ==
**\documentclass{article}**
**\usepackage{graphicx}**
**\begin{document}**
**\centering**
**   \begin{tabular}{c c}**
**  \includegraphics[scale=.41]{pic/1b.png}  &  \includegraphics[scale=.41]{pic/1a.png} \\**
**\end{tabular}**
**  \caption{A boat.}**
**  \label{fig:boat1}**
**\end{figure}**
**Figure \ref{fig:boat1} shows a boat.**
**\end{document}**



== Moving a Latex figure to the left ==

\begin{figure}[htbp]
	**\hspace*{-2cm}**  
	\includegraphics{./results/fig5.eps}
	\caption{Time average}
	\label{fig:f1}
\end{figure}

== Vertical blank spaces ==
//\vspace{5mm} %5mm vertical space//
//\vfill//
 Text at the bottom of the page.
https://www.sharelatex.com/learn/Line_breaks_and_blank_spaces




=== %line spacing after xepersian ===
\makeatletter
\newcommand*{\Computebaselinestretch}[1]{%
  \strip@pt\dimexpr\number\numexpr\number\dimexpr#1\relax*65536/\number\dimexpr\baselineskip\relax\relax sp\relax
}
\makeatother
\linespread{\Computebaselinestretch{1.5cm}}

فقط فاصله خطوط عوض بشه این کافیه    
\linespread{1.5}



====== bibliography ======

\clearpage    % Forcing bibliography to the end 
\medskip
\bibliographystyle{unsrt}
% \bibliographystyle{plain} % such as plain
\bibliography{lib.bib} %such as MyReferences

===== graphic path =====
\graphicspath{ {fig2/} }


===== \begin{array}{lcl} =====
z & = & a \\
& = & a \\
f(x,y,z) & = & x + y + z
\end{array}


====== trim figure ======
https://tex.stackexchange.com/questions/57418/crop-an-inserted-image
 trim={<left> <lower> <right> <upper>}
\includegraphics[trim=50 50 0 50,clip,scale=0.3]{7.eps}


===== Force figure placement in text =====
The short answer: use the “float” package and then the [H] option for your figure.
\begin{figure}[H]



use bibliography in ms word:
use bib2xml.py to convert bib file to xml
read this page for the rest of steps:
http://interfacegroup.ch/how-can-i-use-my-bibtex-library-in-ms-word/

Step 1: Conversion of the .bib-file
As only citations from .xml files can be inserted in Word, the .bib bibliography must be converted from .bib to the Word-compatible xml format. Luckily, JabRef offers the possibility to export your library into an .xml file (File → Export → Files of type: “MS Office 2007 (*.xml)”)

Step 2: Import relevant citations
This newly exported file with all your citations can then be directly imported into Word documents (References → Manage Sources → Browse… ). All entries of the .xml file will then appear in the “master list”. But, before they can be added to the document, the relevant entries must be copied to the “current list”.

Step 3: Inserting citations and bibliography
After importing the bibliography into Word, any reference from the “Current List” can be cited (References → Insert Citation). And, finally, the bibliography can be added to the document (References → Bibliography).




